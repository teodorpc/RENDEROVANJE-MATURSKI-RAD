\documentclass[12pt]{article}

\usepackage[T1,T2A]{fontenc}
\usepackage[english]{babel}
\usepackage[type1]{libertine}
\usepackage{microtype}

\babelprovide[import,main]{serbian-cyrillic}

\newcommand\textenglish[1]%
{\foreignlanguage{english}{\fontencoding{T1}\selectfont#1}}
\newenvironment{english}%
{\begin{otherlanguage}{english}\fontencoding{T1}\selectfont}%
	{\end{otherlanguage}}

\NeedsTeXFormat{LaTeX2e}[1994/06/01]
\ProvidesPackage{pek}

\RequirePackage{amsmath, amsfonts, amssymb, amsthm, braket, bbold, fullpage, tikz, pgfplots, pgffor, empheq, mathtools}
\usetikzlibrary {positioning}
\newcommand{\crtNaH}[1]{\draw (#1,-0.1) node[anchor=north] {#1}  [thick] (#1, -0.1) -- (#1, 0.1);}
\newcommand{\bd}{\textbf}
\newcommand{\evl}[3]{|_{#1=#2}^{#1=#3}}


\newcommand{\graf}[7]{
	
	\begin{scope}
		\clip (-0.1,-0.1) rectangle (#1,#3);
		\draw[->] (-0.1,0) -- (#1,0) node[below] {$#2$};
		\draw[->] (0,-0.1) -- (0,#3) node[left] {$#4$};
		
		
		\clip (0,0) rectangle (#1,#3);
		\draw[scale=1,domain=0:#7,smooth,variable=#5,black] plot ({#5}, {#6});
	\end{scope}

	\foreach \n in {1,...,#1}{
		\draw (\n,-0.1) node[anchor=north] {\n}  [thick] (\n-1, -0.1) -- (\n-1, 0.1);
	}

	\foreach \n in {1,...,#3}{
			\draw (-0.1,\n) node[anchor=east] {\n}  [thick] (-0.1, \n-1) -- (0.1, \n-1);
	}
	

	\draw (0,-0.09) node[anchor=north] {0};
}


\endinput

\title{Рендеровање}
\author{Теодор Ђелић}
\date{Јун 2020}



\begin{document}
	
	\maketitle
	
	\section*{Апстракт}
	У овом раду ћете сазнати шта је компјутерска графика, шта је рендеровање, историјат његовог развоја, , као и различите технике његове реализације. Приказаћемо илустрацију рендеровања кроз пример-апликацију која користи модерну имплементацију OpenGL-а.
	
	\section{Увод}
	
	\subsection{Компјутерска графика}
	Компјутерска графика је једна од подобласти компјутерских наука која се, најпростије речено, бави свим процесима који су везани за приказивање слике на екрану, односно од самог процеса обликовања и стварања података који чувају информације облика и модела за приказ, обраде тих информација и текстура, рендеровање и осветљавање модела, све до дигиталних приказа слика на екрану.
	\subsection{Рендеровање као појам}
	Као што смо споменули, рендеровање је кључан део у процесу приказивања слике на екрану. Описали бисмо рендеровање као процес генерисања слике из неких података помоћу софтверског програма. Наравно, овај опис је апстрактан (знамо и сами да се процес генерисања слике у фотошопу разликује од процеса цртања појединачних фрејмова у игрици), па зато и делимо рендеровање по различитим критеријумима на више подгрупа. За критеријум можемо узети разне факторе (нпр. да ли је жељена фотореалистична слика или не), али постоји један који је изнад свих, односно он је онај примарни те га узимамо за главну поделу, а то је критеријум важности брзине извршавања и по њему делимо рендеровање у две категорије: рендеровање у реалном времену и споро рендеровање (тзв. пре-рендеровање). Нпр. у игрици нам је битно да се овај процес одвија јако брзо како би кориснику могли приказати најскорији приказ његовог видокруга како би остварили илузију флуидности, док када бисмо желели да створимо фотореалистичну слику помоћу некакве симулације светлосних зрака, могли бисмо да приуштимо да се тај процес одвија чак и неколико дана, пошто нам је битан само резултат, а не и време за које се тај резултат добије. Но, наравно, ова подела нам идаље не одређује ништа уже процес рендеровања, односно сам процес рендеровања ми дефинишемо и стварамо по томе шта је наш жељени резултат, и ми све те подгрупе називамо различитим техникама рендеровања. Неке од њих ћу споменути и разрадити у каснијем делу рада.
	
	\tableofcontents
	
\end{document}